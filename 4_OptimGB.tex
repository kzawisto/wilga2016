\section{TEST OF GHOST BUSTER ALGORITHM}

During the step of muon track reconstruction it is common that one physical muon causes a few muon candidates with similar track parameters. The additional candidates are often called \textit{ghosts}. The step of eliminating \textit{ghosts} is called GhostBuster. GhostBuster takes the input set of muon candidates and produces the output set of selected candidates. The algorithm of GhostBuster is based on comparing azimuthal angle of muon candidates. Sometimes two or more physical muons appear in Overlap region per one event. Well working GhostBuster is expected to select the same number of muon candidates as the number of physical muons. For the present article performance of GhostBuster emulator was studied.

In the beginning the input set of candidates is taken and they are sorted by quality. Then all candidates are checked in a loop whether their azimuthal angle is close to the azimuthal angle of candidates which are already in the output set (that is performed in a nested loop). If the angle difference is smaller than veto \textit{window} (set to 5$^\circ$) then the candidate from the input set is recognized as a \textit{ghost} and is not passed to the output set. Consequently input candidate with the best quality is always forwarded to the output set.

Chance for finding and eliminating \textit{ghosts} successfully depends significantly on the width of the veto \textit{window}. To study this dependancy data samples with simulated single muon events were used. In the Fig. \ref{prob_vs_phi} there is a plot of probability of getting certain number of candidates at the GhostBuster output set vs veto \textit{window} width. This plot was done for events with both generated muons propagating into the whole Overlap region, that is with pseudorapidity $0.83 < \eta < 1.24$.
\begin{figure}[t]
\centering
\includegraphics[width=0.7\textwidth]{prob_phi.pdf}
\caption{Results obtained for sample with single muon events. Probability of multiple muon reconstruction on the output of GhostBuster vs $\phi$ \textit{window} width. Points on the left edge of plot refer to \textit{window} width set to 0$^\circ$. }
\label{prob_vs_phi}
\end{figure} 
For the very small values of veto \textit{window} width GhostBuster's performance is degenerated - in most cases single physical muon will be reconstructed as multiple candidates. For 0.5$^\circ$\textrm{ }and 1$^\circ$\textrm{ }GhostBuster is clearly much more effective and for 5$^\circ$\textrm{ }probability of receiving single candidate on the output set is bigger than 99.9\%. 

Good \textit{ghosts} selection has however its price. For high transverse momenta GhostBuster decreases efficiency of the $\mu^{+}\mu^{-}$ pairs reconstruction. This effect was studied on data samples with simulated single J/$\Psi \rightarrow \mu^{+}\mu^{-}$ decay events. In the Fig. \ref{divmom} there is a plot of efficiency of the $\mu^{+}\mu^{-}$ pairs reconstruction vs J/$\psi$ generated transverse momentum. This plot was done for events with both generated muons propagating into the middle Overlap region, that is with pseudorapidity $0.9 < \eta < 1.15$. For J/$\psi$ transverse momentum bigger than 40 GeV/c efficiency drops from $(92.4\pm0.5)\%$ (when cut on azimuthal angle is disabled) to $(86.2\pm0.6)\%$ (when 5$^\circ$\textrm{ }cut is used). This is a consequence of fact that for high J/$\psi$ momenta the decay resulting in muons propagating into narrow $\phi$ \textit{window} is more probable.
\begin{figure}[t]
\centering
\includegraphics[width=0.7\textwidth]{efficiency_GB_5.pdf}
\caption{Efficiency of reconstructing two muon candidates vs J/$\psi$ transverse momentum.}
\label{divmom}
\end{figure} 
