\section{TEST OF GHOST BUSTER ALGORITHM}

Na etapie rekonstrukcji torów często się zdarza, że jeden fizyczny mion zostanie zrekonstruowany jako kilka mionów o zbliżonych parametrach toru.  Istotnym zadaniem trygera jest wyeliminowanie takich duplikatów, które określane są często mianem duchów. Dokonuje się to poprzez odrzucanie torów o podobnym kącie azymutalnym. Na podstawie analizy danych z symulacji Monte Carlo zbadano jak zmienia się efektywność eliminowania duplikatów w zależności od przedziału dyskryminującego w kącie azymutalnym (rys. lewy). Standardowy dobór przedziału dyskryminującego (5 stopni) pozwala odrzucić ponad 99\% duplikatów. Ceną za skuteczną selekcję duchów jest zmniejszona efektywność rekonstrukcji par µ+µ- w obszarze wysokich pędów poprzecznych. Efekt ten przeanalizowano na wysymulowanej próbce danych z rozpadem pojedynczego J/Ψ → µ+µ-. Na rys. prawym przedstawiono efektywność rekonstrukcji par w funkcji pędu poprzecznego J/Ψ. Wykres wykonano przy włączonym eliminowaniu duplikatów przy standardowym przedziale dyskryminacyjnym. Dla pędów poprzecznych J/Ψ powyżej 50 GeV/c efektywność rekonstrukcji par spada z (91.8±0.7)\% do (80.8±1.0)\%.
