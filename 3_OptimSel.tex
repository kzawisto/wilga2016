\section{OPTIMISATION OF EVENT SELECTION}  

Podstawowym zadaniem trygera mionowego jest rekonstrukcja mionów w zdarzeniu w celu dokonania  tzw. cięcia, umożliwiającego wybranie przypadków zawierających wysokoenergetyczne miony. Ciecie polega na odrzuceniu przypadków z mionami o pędzie poprzecznym poniżej zadanego progu. Na rysunku poniżej przedstawiono efektywność trygera w funkcji pędu poprzecznego mionu dla przykładowego cięcia pędowego o progu 16 GeV/c. Efektywność jest tu definiowana jako iloraz liczby mionów nieodrzuconych do wszystkich mionów. Idealna krzywa efektywności byłaby krzywą schodkową zmieniającą efektywność z 0 na 1 przy wartości progowej pędu. Na rysunku wskazano dwie, będące wynikiem pracy, opcje selekcji kandydatów mionowych rekonstruowanych przez OMTF (actual, improved), oraz porównano je z wyznaczoną najlepszą możliwą dla danego algorytmu (best). Na kolejnym rysunku wykreślono stosunek krzywych efektywności uzyskanych przez warianty selekcji. Algorytm zoptymalizowany uwzględnia zróżnicowanie efektywności poszczególnych warstw referencyjnych oraz waży wartości PDF w zależności od typu detektora.                                                                          

