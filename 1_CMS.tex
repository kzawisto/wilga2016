\section{COMPACT MUON SOLENOID A SELEKCJA PRZYPADKOW}
\label{sec:1_CMS}

Compact Moun Solenoid jest jednym z eksperymentów przy Wielkim Zderzaczu Hadronów w Europejskiej Organizacji Badań Jądrowych (CERN). W pierwszej fazie działania CMS odkrył (wraz z eksperymentem ATLAS) bozon Higgsa. Ze względu na wysoką częstość zderzeń proton-proton oraz ograniczoną możliwość zapisu przypadków konieczny jest system wyzwalania detektora (tryger). Jest on odpowiedzialny za wstępną selekcję przypadków do późniejszej analizy. Podstawowe cechy, którymi powinien się charakteryzować układ wyzwalania to wysoka efektywność wyzwalania na przypadki interesujące oraz czystość, czyli skuteczność odrzucania pozostałych przypadków. Po pierwszej fazie działania, ze względu na zwiększenie świetlności i energii akceleratora system wyzwalania jest modernizowany. Grupa Warszawska zajmuje się w CMS trygerem mionowym pier­w
szego poziomu. W ramach obecnej modernizacji budujemy nowy tryger w tzw. obszarze pośrednim detektora, gdzie trajektorie mi­onów przecinają detektory za­równo beczki 
(obszar centralny) jak i wiek (przód i tył).
