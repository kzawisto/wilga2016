\section{COMPACT MUON SOLENOID AND EVENT SELECTION}
\label{sec:1_CMS}
Compact Muon Solenoid (CMS)~\cite{CMS-JINST} is an experiment at the Large Hadron Collider (LHC) at European Organisation for Nuclear Research (CERN). During the Run-I of the LHC (taking data from 2009 to 2013) the CMS and ATLAS collaborations discovered Higgs boson (that was officially announced in July 2012). Because of the very high rate of proton-proton collisions inside the CMS detector and limited ability of gathering data it is necessary to use trigger system which automatically selects part of the events for further analysis. Selection is done in two steps - in the Level-1 Trigger and the High Level Trigger. One of the crucial components of the Level-1 Trigger is the Level-1 Muon Trigger - its development is currently one of the main tasks of the CMS Warsaw Group. Good muon trigger is expected to reconstruct muon tracks efficiently and select events which are interesting for current studies. 

There are three types of muon chambers in the CMS detector: Drift Tubes (DTs), Cathode Strip Chambers (CSCs) and Resistive Plate Chambers (RPCs). In the old muon trigger there were three parallel systems providing muon candidates. Each system was using data from different type of muon chambers. The Level-1 Muon Trigger is being upgraded for the LHC Run-II (taking data since 2015) and the new strategy is introduced. It is based on track finders working in different geometrical regions of the CMS detector: the Barrel, the Endcap and the Overlap. In upgraded trigger system track finders should combine signals from all available types of chambers.

