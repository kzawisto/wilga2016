\section{COMPACT MUON SOLENOID AND EVENT SELECTION}
\label{sec:1_CMS}

The Compact Muon Solenoid(CMS) experiment is a general purpose particle detector in Large Hadron Collider (LHC) at European Organization for Nuclear Research (CERN).
In the first phase of research in CMS there was discovered the Higgs boson in 2012 (as well as in ATLAS experiment).
It is necessary to select events because of very high rate of proton-proton collisions and limited ability of saving produced data.
This preliminary selection of events for futher analysis is made by system called the muon trigger system.
For good operating, trigger should be as effective as it is possible with purity kept in a high level.
This mean, it should accept all intresting muons and reject the rest.
After the first operational LHC run, all of the trigger system are modernized because of increase of the luminosity and the collison energy in LHC.
The CMS Warsaw Group is working on Level-1 Muon Trigger for CMS.
In the current modernisation phase we are creating a new trigger for the overlap range, that is the area that muons cross detectors in both barrel and endcap parts.

