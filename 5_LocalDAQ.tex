\section{SYSTEM TESTS WITH LOCAL DATA}

For further (offline) event reconstruction and data analysis CMS is 
equipped with Data Acquisition
System (DAQ), which saves events selected by trigger. Regardless of
it, OMTF introduced ability to read data directly from memory buffers.
This readout set was called Local Data Acquisition System (Local DAQ)
and it is a kind of \textit{spy module}, which means it allows to snapshot
some raw data collected by OMTF  for further prompt feedback analyses.

Local DAQ returns 3 steps of data acquisition and analysis - raw hits
collected by detectors (AlgoHits), data after selection by hardware-built-in
algorithm (AlgoMuons) and data after ghost-busting (CandMuons). There
is too much data being processed by hardware at one point of time,
so Local DAQ reads it with limited bandwidth.

Data collected from Local DAQ can be analysed to check angular distribution
of hits from all types of detector and out-coming muons. 
It is expected to be roughly uniform, with some
fluctuations in the chamber overlapping region due to chamber acceptance. 
If expectations are
met, it indicates that detectors work properly and give CMS statistically
correct data.

Other important things to check are correlations of hits between chambers.
Due to geometry of OMTF, some chambers are placed directly next to
others (RPC and DT in barrel region or RPC and CSC in endcap region).
This allows to quickly check whether detectors work properly by data correlation
analysis. If AlgoMuons data is included, hardware algorithm work may
be checked either.

The next step in local data analysis is to compare results of hardware-built-in
algorithm with outcome of emulation of algorithm work. To do this the events
from simulation (test pulses) are injected into electronics, processed and
read-out. The results must exactly match the expectations.

The Local DAQ and simplified analysis performed with collected data played 
a key role in initial commissioning of OMTF. The readout of memory buffers 
will be further used for online OMTF monitoring.
