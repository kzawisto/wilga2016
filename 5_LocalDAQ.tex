\section{SYSTEM TESTS WITH LOCAL DATA}
For further reconstruction analysis CMS was equipped with Data Acquisition
System (DAQ), which saves cases selected by detector. Regardless of
it, OMTF introduced ability to read data directly from memory buffers.
This readout set was called Local Data Acquisition System (Local DAQ)
and it is a kind of \textit{spy module}, which means it allows to snapshot
some raw data collected by detectors for further hits correlation
or system feedback analysis.

Local DAQ returns 3 steps of data acquisition and analysis - raw hits
collected by detectors (AlgoHits), data after selection by hardware-built-in
algorithm (AlgoMuons) and data after ghostbusting (CandMuons). There
is too much data being processed by hardware at one point of time,
so Local DAQ reads it with bounded bandwith.

Data collected from Local DAQ can be analysed to check angular distribution
of hits in the barrel region. It is expected to be uniform with some
fluctuations in the chamber overlapping region. If expectations are
met, it means that detectors work properly and give CMS statistically
correct data.

Other important things to check are correlations of hits between chambers.
Due to geometry of OMTF, some chambers are placed directly next to
others (RPC and DT in barrel region or RPC and CSC in endcap region).
This allows to check whether detectors work properly by data correlation
analysis. If AlgoMuons data is included, hardware algorithm work may
be checked either.

The next step in local data analysis is to compare results of hardware-built-in
algorithm with outcome of emulation of algorithm work. There are plenty
of features to check such as muon energy, muon momentum, hit phi angle
and so on. Such analysis will help to upgrade algorithm and in effect
total efficiency of OMTF.
