\section{THE MUON SELECTION IN THE DETECTOR OVERLAP REGION}

The algorithm fo the Overlap Muon Track Finder (OMTF) was designed to reconstruct muon tracks in the Overlap region at the Level-1 Muon Trigger. The algorithm is based on comparing measured hits with precomputed track patterns, so-called Golden Patterns (GPs). Each GP corresponds to the muon track with a certain charge (i.e. $\mu^{+}$, $\mu^{-}$) and transverse momentum. GP comprises information about track's average azimuthal angle bending $\Delta \phi_{mean}$ between reference layer and every other layer caused by CMS magnetic field. There are also included Probability Density Function (PDF) values of possible deviations from $\Delta \phi_{mean}$ values (due to stochastic effects: multiple scattering and energy losses). 

Constant set of 8 reference layers is foreordained. In the beginning of the algorithm, up to 4 reference hits in reference layers are chosen (among all physical hits in the event) to make possible finding multiple muon candidates. Then for every signal from the event there is calculated the azimuthal angle difference $\Delta \phi_{i}$ between certain reference hit and each hit laying in acceptable logic region. In order to compare this with certain Golden Pattern algorithm calculates difference between average azimuthal angle track bending for given GP and actual track bending ($\phi_{dist} = \Delta\phi_{mean} - \Delta\phi_{i}$)% (Fig. \ref{algorithm})
 for every hit. For each layer hit with the smallest $\phi_{dist}$ value is chosen and corresponding PDF value is taken. If the PDF value is bigger than 0, the processed layer is counted as an \textit{active layer}. Subsequently the PDF values from all layers are added up. In the end for each reference hit there is chosen one Golden Pattern fitted best to event hits. The basic criterion for that choice is the number of \textit{active layers} corresponding to each GP - the pattern with more \textit{active layers} is favoured. If that number is the same for a few GPs then the pattern with the biggest sum of PDF values is selected. As this procedure is followed for up to 4 reference hits, on the output there are up to 4 candidates in the Overlap region per one event.

Described algorithm was developed on the grounds of Monte Carlo simulations and is implemented in dedicated electronics as a FPGA firmware.

\cite{OMTF-Wilga2014}
