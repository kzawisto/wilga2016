\section{THE MUON SELECTION IN A DETECTOR OVERLAP REGION}

Algorytm OMTF (Overlap Muon Track Finder) został zaprojektowany z myślą o rekonstrukcji torów mionowych w obszarze pośrednim na poziomie trygera pierwszego poziomu. OMTF bazuje na porównywaniu sygnału z detektora z gotowymi wzorcami torów (Golden Patterns). Każdy taki wzorzec reprezentuje mion o określonym znaku (µ+, µ-) i pędzie poprzecznym. Golden Pattern zawiera informacje o krzywiźnie toru mionu wewnątrz detektora i o średnim gięciu w kącie azymutalnym ΔΦmean pomiędzy warstwą referencyjną a każdą inną warstwą. Ponadto Golden Pattern zawiera informację o funkcji gęstości prawdopodobieństwa (PDF) możliwego odchylenia od wartości ΔΦmean.
Odgórnie zostaje narzucony zbiór ośmiu warstw referencyjnych. Na początku wybierane jest trafienie referencyjne. Następnie dla każdego sygnału ze zdarzenia obliczana jest różnica w kącie azymutalnym ΔΦi pomiędzy każdym trafieniem i danym trafieniem referencyjnym. Aby przyrównać to do wzorca toru algorytm oblicza dla każdego trafienia różnicę między bieżącym gięciem toru i średnim gięciem toru zapisanym w Golden Patternie (Φdist = ΔΦmean - ΔΦi). W każdej warstwie trafienie z najmniejszą wartością przypisanego Φdist jest wybrane i odczytana zostaje wartość PDF. Następnie obliczona zostaje suma PDF po wszystkich warstwach. Jeśli do wybranego trafienia przypisane jest PDF > 0, to wybrana warstwa jest liczona jaka warstwa aktywna. Ostatecznie, do sygnału zostaje dopasowany taki Golden Pattern, któremu odpowiada największa liczbę aktywnych warstw i największa suma PDF. Powyższe działanie jest powtarzane dla maksymalnie czterech trafień referencyjnych, które mogą się pojawić w ośmiu wybranych płaszczyznach detektora. Algorytm został opracowany na podstawie komputerowych symulacji Monte Carlo i zaimplementowany w dedykowanej elektronice jako firmware układów FPGA.

\cite{OMTF-Wilga2014}
