\begin{abstract}

Present article is a description of the authors contribution in upgrade and analysis of performance of the Level-1 Muon Trigger of the CMS experiment. The authors are students of University of Warsaw and Gdansk University of Technology. They are collaborating with the CMS Warsaw Group. This article summarise students' work presented during the Students session during the Workshop XXXVIII-th IEEE-SPIE Joint Symposium Wilga 2016.

In the first section the CMS experiment is briefly described and the importance of the trigger system is explained. There is also shown basic difference between old muon trigger strategy and the upgraded one.

The second section is devoted to Overlap Muon Track Finder (OMTF). This is one of the crucial components of the Level-1 Muon Trigger. The algorithm of OMTF is described.

In the third section there is discussed one of the event selection aspects - cut on the muon transverse momentum $p_{T}$. Sometimes physical muon with $p_{T}$ bigger than a certain threshold is unnecessarily cut and physical muon with lower $p_{T}$ survives. To improve $p_{T}$ selection modified algorithm was proposed and its performance was studied.

One of the features of the OMTF is that one physical muon often results in several muon candidates. The Ghost-Buster algorithm is designed to eliminate surplus candidates. In the fourth section this algorithm and its performance on different data samples are discussed.

In the fifth section Local Data Acquisition System (Local DAQ) is briefly described. It supports initial system commissioning. The test done with OMTF Local DAQ are described.  

In the sixth section there is described development of web application used for the control and monitoring of CMS electronics. The application provides access to graphical user interface for manual control and the connection to the CMS hierarchical Run Control.
\end{abstract}
